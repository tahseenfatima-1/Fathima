\documentclass{article}
\usepackage{amsmath}
\usepackage{gvv}
\title{\underline{\textbf{jee2020-paper1}}}
\date{}

\begin{document}
\maketitle
\begin{enumerate}

\item If $g(x) = x^2 + x - 1$ and $(g \circ f)(x) = 4x^2 - 10x + 5$, then $f\brak{\frac{5}{4}}$ is equal to:  
\begin{enumerate}
    \item $\frac{3}{2}$
    \item $-\frac{1}{2}$
    \item $-\frac{3}{2}$
    \item $\frac{1}{2}$
\end{enumerate}

\item If $\operatorname{Re}\brak{\frac{z - 1}{2z + i}} = 1$, where $z = x + iy$, then the point $(x, y)$ lies on a:

\begin{enumerate}
    \item Circle whose centre is at $\brak{-\frac{1}{2}, -\frac{3}{2}}$.
    \item Circle whose diameter is ${\frac{\sqrt{5}}{2}}$.
    \item Straight line whose slope is ${\frac{3}{2}}$.
    \item Straight line whose slope is ${-\frac{2}{3}}$.
\end{enumerate}


\item Five numbers are in arithmetic progression A.P., whose sum is 25 and product is 2520. If one of these five numbers is $-\frac{1}{2}$, then the greatest number among them is:  

\begin{enumerate}
    \item $\frac{21}{2}$
    \item $27$
    \item $16$
    \item $7$
\end{enumerate}

\item If 
\[
y(\alpha) = \sqrt{2 \left( \frac{\tan \alpha + \cot \alpha}{1 + \tan^2 \alpha} \right) + \frac{1}{\sin^2 \alpha}}, \quad \alpha \in \left( \frac{3\pi}{4}, \pi \right),
\]
then 
\[
\frac{dy}{d\alpha} \text{ at } \alpha = \frac{5\pi}{6} \text{ is:}
\]

\begin{enumerate}
    \item $4$
    \item $-\frac{1}{4}$
    \item $\frac{4}{3}$
    \item $-4$
\end{enumerate}

\item Let $\alpha$ be a root of the equation $x^2 + x + 1 = 0$, and let the matrix $A = \frac{1}{\sqrt{3}} 
\begin{bmatrix}
1 & 1 & 1 \\
1 & \alpha & \alpha^2 \\
1 & \alpha^2 & \alpha^4
\end{bmatrix}$, then the matrix $A^{31}$ is equal to:  
\begin{enumerate}
    \item $A^3$
    \item $A$
    \item $A^2$
    \item $I3$
\end{enumerate}

\item If $y = mx + 4$ is a tangent to both the parabolas $y^2 = 4x$ and $x^2 = 2by$, then $b$ is equal to:  

\begin{enumerate}
    \item $128$
    \item $264$
    \item $128$
    \item $-32$
\end{enumerate}

\item If the distance between the foci of an ellipse is $6$ and the distance between its directrices is $12$, then the length of its latus rectum is:  

\begin{enumerate}
    \item $\sqrt{3}$
    \item $2\sqrt{3}$
    \item $3\sqrt{2}$
    \item $\frac{3}{\sqrt{2}}$
\end{enumerate}

\item An unbiased coin is tossed 5 times. Suppose that a variable $X$ is assigned the value $k$ when $k$ consecutive heads are obtained for $k = 3, 4, 5$, otherwise $X$ takes the value $1$. Then the expected value of $X$ is:  

\begin{enumerate}
    \item $\frac{3}{16}$
    \item $-\frac{3}{16}$
    \item $\frac{1}{8}$
    \item $-\frac{1}{8}$
\end{enumerate}

\item The area of the region enclosed by the circle $x^2 + y^2 = 2$ which is not common to the region bounded by the parabola $y^2 = x$ and the straight line $y = x$, is:  

\begin{enumerate}
    \item $\frac{1}{3}(12{\pi} - 1)$
    \item $\frac{1}{6}(12\pi - 1)$
    \item $\frac{1}{6}{(24{\pi} -{1})}$
    \item $\frac{1}{3}(l6\pi - 1)$
\end{enumerate}


\item Let $x^k + y^k = a^k$, $(a, k > 0)$ and $\frac{dy}{dx} + ( \frac{y}{x} )^{\frac{1}{3}} = 0$, then $k$ is:

\begin{enumerate}
        \item $\frac{1}{2}$
        \item $\frac{1}{3}$
        \item $\frac{2}{3}$
        \item $\frac{4}{3}$
\end{enumerate}


\item If $y = y(x)$ is the solution of the differential equation $e^y\cbrak{\frac{dy}{dx} - 1} = e^x$ such that $y(0) = 0$, then $y(1)$ is equal to:  

\begin{enumerate}
    \item $2 + \log_e 2$
    \item $2e$
    \item $log_e 2$
    \item $1 + \log_e 2$
\end{enumerate}

\item Total number of 6 digit numbers in which only and all the five digits 1,3,5,7 and 9 appers is:

\begin{enumerate}
	\item $\frac{5}{2}(6!)$
	\item $5^6$
	\item $\frac{1}{2}(6!)$
	\item $6!$	
\end{enumerate}

\item Let $P$ be a plane passing through the points $(2, 1, 0)$, $(4, 1, 1)$, and $(5, 0, 1)$, and let $R$ be the point $(2, 1, 6)$. Then the image of $R$ in the plane $P$ is:

\begin{enumerate}
    \item $(6, 5, -2)$
    \item $(4, 3, 2)$
    \item $(3,4,-2)$
    \item $(6, 5, 2)$
\end{enumerate}

\item A vector $\vec{a} = \alpha \hat{i} + 2\hat{j} + \beta \hat{k}$
($\alpha, \beta \in \mathbb{R}$) lies in the plane of the vectors 
$\vec{b} = \hat{i} + \hat{j}$ and 
$\vec{c} = \hat{i} - \hat{j} + 4\hat{k}$. 
If $\vec{a}$ bisects the angle between $\vec{b}$ and $\vec{c}$, then:

\begin{enumerate}
    \item $\vec{a} \cdot \hat{i} + 1 = 0$
    \item $\vec{a} \cdot \hat{i} + 3 = 0$
    \item $\vec{a} \cdot \hat{k} + 4 = 0$
    \item $\vec{a} \cdot \hat{k} + 2 = 0$
\end{enumerate}


\item If $f(a + b + 1 - x) = f(x)$ for all $x$, where $a$ and $b$ are fixed positive real numbers, then
\[\frac{1}{a + b} \int_a^b x \big(f(x) + f(x+1)\big) \, dx\]
is equal to:

\begin{enumerate}
    \item $\int_{a+1}^{b+1} f(x) \, dx$
    \item $\int_{a+1}^{b+1} f(x+1) \, dx$
    \item $\int_{a-1}^{b-1} f(x+1) \, dx$
    \item $\int_{a-1}^{b-1} f(x) \, dx$
\end{enumerate}

\item Let the function $f : [-7, 0] \to \mathbb{R}$ be continuous on $[-7, 0]$ and differentiable on $(-7, 0)$. If $f(-7) = -3$ and $f'(x) \leq 2$ for all $x \in (-7, 0)$, then for all such functions $f$, $f(-1) + f(0)$ lies in the interval:

\begin{enumerate}
    \item $[-6, 20]$
    \item $(-\infty, 20]$
    \item $(-\infty, 11]$
    \item $[-3, 11]$
\end{enumerate}

\item If the system of linear equations
\begin{align*}
    2x + 2ay + az &= 0, \\
    2x + 3by + bz &= 0, \\
    2x + 4cy + cz &= 0,
\end{align*}
where $a, b, c$ are non-zero and distinct, has a non-zero solution, then:

\begin{enumerate}
    \item $a, b, c$ are in A.P.
    \item $a + b + c = 0$
    \item $a, b, c$ are in G.P.
    \item $\frac{1}{a}, \frac{1}{b}, \frac{1}{c}$ are in A.P.
\end{enumerate}


\item Let $\alpha$ and $\beta$ be two real roots of the equation  \[(k+1) \tan^2 x - \sqrt{2} \cdot \lambda \tan x = (1 - k),\]where $k (\neq -1)$ and $\lambda$ are real numbers. If\[\tan^{-1} (\alpha + \beta) = 50,\]
then the value of $\lambda$ is:

\begin{enumerate}
    \item $5$
    \item $10$
    \item $5\sqrt{2}$
    \item $10\sqrt{2}$
\end{enumerate}

\item The logical statement $(p \Rightarrow q) \land
	(q \Rightarrow \neg p)$

\begin{enumerate}
	\item $p$
	\item $q$
        \item $~p$
        \item $~q$
\end{enumerate}

\item The greatest positive integer $k$, for which $49^k + 1$ is a factor of the sum 
$19^{125}× + 49^{124} + \dots + 49^2 + 49 + 1,$
is:

\begin{enumerate}
    \item 32
    \item 60
    \item 63
    \item 65
\end{enumerate}

\item
\[
\lim_{x \to 2} \frac{3^3 + 3^{3-x} - 12}{3^{-x/2} - 3^{1-x}}
\]

is equal to \underline{\hspace{2cm}}.

\item If the variance of the first $n$ natural numbers is 10 and the variance of the first $m$ even natural numbers is 16, then $m+n$ is equal to \underline{\hspace{2cm}}.

\item If the sum of the coefficients of all even powers of $x$ in the product(1+x+ $x^2$ + \dots + $x^{2n}$) (1-x+ $x^2$ - $x^3$ + \dots + $x^{2n}$)

\item Let $A(1, 0)$, $B(6, 2)$, and $C(\frac{3}{2},6)$ be the vertices of a triangle $ABC$. If $P$ is a point inside the triangle $ABC$ such that the triangles $APC$, $APB$, and $BPC$ have equal areas, then the length of the line segment $PQ$, where $Q$ is the point $(-\frac{7}{6}, -\frac{1}{3})$, is \underline{\hspace{2cm}}


\item Let $S$ be the set of points where the function
$f(x) = |2 - |x - 3||$, $\quad x \in {R}$ is not differentiable. Then,$\sum_{x \in S}$ $f(f(x))$
is equal to \underline{\hspace{1cm}}.



\end{enumerate}

\end{document}
